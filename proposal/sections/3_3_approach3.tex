% !TEX root=../proposal.tex

\section{Approach 3: }
We will use evolutionary techniques to brute force the problems.
The main idea is to see if the quantity of updates will prevail over the quality over updates.
We are equipped to solve this problem as we have access to a cluster with over $300$ nodes with $4$ cores per node.
This yields over $1200$ cores total, which is on the same order of magnitude as the cluster sizes used for state of the art research~\cite{salimans2017evolution}.
There are many metrics worth considering here, but the main ones will measure learning progress per unit time.
This is because for every update a traditional algorithm will due, the evolutionary strategy can do thousands.

One avenue worth exploring is whether this sort of problem can be optimized and deployed effectively over GPUs.
Evolutionary algorithms are embarassingly parallel, but it's not necessarily true that they will run well on GPUs.
Furter, it’s may be worth investigating how parameter sharing impacts an evolutionary search.
Sharing parameters decreases the complexity of the learning task, however it increases the required communication required in the system.
Allowing parameters to grow stale may help counteract some of these negative effects~\cite{cui2014exploiting}.
\label{sec:direction3}



