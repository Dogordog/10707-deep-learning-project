% !TEX root=../report.tex

\section{Environment and Applications}
\label{sec:application}

In this project, we use the OpenAI Gym framework~\cite{gym.openai16}, which
generates virtually infinite amount of data for training and testing our
neural network. This framework includes various gym environments that simulate
various reinforcement learning settings. Each gym environment defines the
\emph{state space} (i.e., all possible states of the environment) and the
action space (i.e., all possible actions of the agent). Note that a gym
environment is turn-based, i.e., the agent can only take one action per time
step to influence the state at the next time step. In each time step, the
environment computes the next state and the immediate reward for each agent,
and decide if the episode is finished or not. These pieces of information
(i.e., the immediate reward and the episode finished signal) are used by each
agent to decide the best action in the next step. We refer the readers to
Section~\ref{sec:background:problem} for a detailed explanation of a classical
reinforcement learning problem setting.

\subsection{The Predator-Pray Environment}

We develop a modified version of the \emph{``predator-pray''} scenario
in the ``multi-agent particle'' environment suite developed by prior
work~\cite{lowe2017multi, mordatch2017emergence}, and train our
neural-network-based agents. In the original classic predator-prey game, $N$
slower \emph{cooperating agents} must chase the faster \emph{adversary}
around a randomly
generated environment with $L$ large landmarks impeding the way. Each time the
cooperative agents collide with an adversary, the agents are rewarded
while the adversary is penalized. Agents observe the relative positions and
velocities of the adversary, and the positions of the landmarks. We introduce
the state space, action space, the reward function, and our modifications to
the scenario.

\paragraph{State Space.} In the predator-pray scenario, each agent can only
observe its own state (i.e., $state[i]$). At each step, the environment
returns these states concatenated into a list, which is then distributed to
each individual agent accordingly. The $state[i]$ of each agent contains 
(1)~the velocity and the location of itself in the x- and y-coordinate,
(2)~the relative x- and y-coordinate of two landmarks,
(3)~the relative x- and y-coordinate of all other agents, and
(4)~the x- and y-directional velocity of of the escaping adversary.

\paragraph{Action Space.} At each step, each agent can take an action (i.e.,
$action[i]$) to move itself. Similar to the sate, all agents' actions are
concatenated into a list and send to the environment. The $action[i]$ of each
agent contains a no-action signal, and the movements towards four directions 
(i.e., right, left, down, or up). In the discrete action space used for DQN,
the action is simply an one-hot encoding of the five possible actions, where
each action has a fixed amplitude. In continuous action space used for DDPG
and MADDPG, the environment computes the amplitude of movement along the
x-axis (i.e., $right - left$) and the y-axis (i.e., $up - down$).

\paragraph{Reward Function.} In the original environment, all agents receive
a fixed positive reward if any agent has collided into the adversary. If such
collision happens, the same amount of negative reward is taken out of the
adversary. In order to keep the game visible on the screen and to reduce the
state space, the adversary also receives negative reward when it goes out of
the computer screen.

\paragraph{Our Modifications.} First, we modify the reward function to make
the scenario more interesting. In addition to the reward function in the
original version, we added positive reward to the adversary for how far it
is to the agents, measured in l2 distance. Similarly, a negative reward is add
to the agent for how far it is to the adversary.
Second, we modify the original predator-pray
scenario to test different number of agents and adversaries. We included
results for (1)~1 agent vs. 1 adversary, (2)~1 agent vs. 2 adversaries, and 
(3)~2 agents vs. 1 adversary. Third, we made the scenario into an episodic
game. The episode never ends in the original version. The episode now ends
when any agent or adversary goes out of the screen.

